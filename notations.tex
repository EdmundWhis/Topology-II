% 用在这一学科里的单独定义(不用在其他 notation 里)

\newcommand{\RP}{\mathbb{RP}}
\newcommand{\CP}{\mathbb{CP}}

%特殊字体定义

\newcommand{\category}[1]{\ensuremath{\mathsf{#1}}}


%运算符定义

\DeclareMathOperator{\rad}{rad}                         % 指弧度
\DeclareMathOperator{\diam}{diam}                       % 指直径
\DeclareMathOperator{\fin}{fin}                         % 指有限子集全体
\DeclareMathOperator{\esssup}{ess\,sup}                 % 指本性上确界
\DeclareMathOperator{\conv}{Conv}                       % 指凸包
\DeclareMathOperator{\Span}{span}                       % 因为\span已经在宏中定义, 这里使用大写的\Span来表示线性张成
\DeclareMathOperator{\cont}{Cont}                       % 表示函数的连续点
\DeclareMathOperator{\diag}{diag}                       % 指对角化的算子或对角矩阵
\DeclareMathOperator{\codim}{codim}                     % 指余维数
\DeclareMathOperator{\convba}{Convba}                   % 指函数的凸平衡包
\DeclareMathOperator{\supp}{supp}                       % 指函数的支撑集

%特殊的需要正体的表达

\newcommand{\me}{\ensuremath{\mathrm{e}}}               % 指自然底数e
\newcommand{\imag}{\mathrm{i}}                          % 指虚数单位i
\newcommand{\1}{\mathds{1}}                             % 指单位向量1或单位矩阵I
\newcommand{\id}{\mathrm{id}}                           % 指恒等映射 identity
\renewcommand{\Re}{\mathrm{Re\,}}                       % 重定义实部 real part
\renewcommand{\Im}{\mathrm{Im\,}}                       % 重定义虚部 imaginary part
\newcommand{\sgn}{\mathrm{sgn}\,}                       % 符号数 sign number
\newcommand{\diff}{\,\mathrm{d}}                        % 微分算子
\renewcommand{\rank}{\ensuremath{\mathrm{rank}\,}}        % 矩阵或算子的秩 rank
\newcommand{\Graph}{\ensuremath{\mathrm{Graph}\,}}      % 映射的图像
\newcommand{\ord}{\ensuremath{\mathrm{ord}\,}}          % 群的阶数(群论)、整数的阶(数论)
\newcommand{\Aut}{\mathrm{Aut}\,}                       % 全体自同构
\newcommand{\Inn}{\mathrm{Inn}\,}                       % 全体内自同构
\newcommand{\End}{\mathrm{End}\,}                       % 全体自同态
\renewcommand{\char}{\mathrm{char}\,}                   % 域的特征
\newcommand{\lcm}{\mathrm{lcm}\,}                       % 最小公倍数 least common multiple
\newcommand{\im}{\mathrm{im}\,}                         % 映射的像
\newcommand{\Mat}{\mathrm{Mat}}                         % 用于矩阵全体 Mat_n(F)
\newcommand{\Sp}{\mathrm{Sp}\,}
\newcommand{\Bd}{\mathrm{Bd}\,}
\newcommand{\St}{\mathrm{St}\,}
\newcommand{\sd}{\mathrm{sd}}
\newcommand{\mesh}{\mathrm{mesh}\,}
\newcommand{\coker}{\mathrm{coker}\,}
\newcommand{\Hom}{\mathrm{Hom}}
\newcommand{\Ext}{\mathrm{Ext}}

% 需要加粗的表达

\newcommand{\clos}{\mathbf{clos}}                       % 闭包


% 嵌套结构
\newcommand{\Star}[1]{#1^{*}}                                           % 带星号的元素(用于对合或伴随)
\newcommand{\set}[1]{\ensuremath{\left\{ #1 \right\}}}                  % 自适应大小的大括号
\renewcommand{\abs}[1]{\ensuremath{\left| #1 \right| }}                   % 自适应大小的绝对值
\renewcommand{\norm}[1]{\ensuremath{\left\| #1 \right\|}}                 % 自适应大小的范数
\newcommand{\tabs}[1]{\ensuremath{\lvert #1\rvert}}                     % 强制一行大小的绝对值
\newcommand{\tnorm}[1]{\ensuremath{\lVert #1\rVert}}                    % 强制一行大小的范数
\newcommand{\tlrangle}[1]{\ensuremath{\langle #1 \rangle}}              % 强制一行大小的尖括号
\newcommand{\Babs}[1]{\ensuremath{\Big| #1 \Big| }}                     % 强制Big大小的绝对值(用于只有下标的巨算符)
\newcommand{\Bnorm}[1]{\ensuremath{\Big\| #1 \Big\|}}                   % 强制Big大小的范数(用于只有下标的巨算符)
\newcommand{\Bset}[1]{\ensuremath{\Big\{ #1 \Big\}}}                    % 强制Big大小的大括号
\newcommand{\Blrangle}[1]{\ensuremath{\Big\langle #1 \Big\rangle}}      % 强制Big大小的尖括号
\newcommand{\lrangle}[1]{\left\langle #1 \right\rangle}                 % 自适应大小的尖括号(用于内积等)
\newcommand{\seq}[2][n]{\ensuremath{#2_{1}, #2_{2}, \dots, #2_{#1}}}    % 用于生成x_1,x_2,...,x_n形式的列
\newcommand{\baro}[1]{\overline{#1}}                                    % 生成长上横线
\renewcommand{\emph}[1]{\textbf{#1}}                                    % 用于标记关键字

% 常用符号的简化定义

\newcommand{\degree}{\ensuremath{^{\circ}}}                             % 指角度
\newcommand{\sm}{\ensuremath{\setminus}}                                % 指差集
\newcommand{\weakto}{\ensuremath{\overset{w.}{\longrightarrow}}}        % 指弱收敛 weak convergent to
\newcommand{\sweakto}{\ensuremath{\overset{\Star{w.}}{\longrightarrow}}}% 指弱*收敛 weak-star convergent to
\newcommand{\norsub}{\triangleleft}                                     % 指正规子群或理想
\newcommand{\tran}{\ensuremath{^\mathtt{T}}}                            % 指矩阵的转置

% 使用 mathtt 字族的符号定义

\newcommand{\GL}{\ensuremath{\mathtt{GL}}}              % 一般线性群 general linear group
\newcommand{\SL}{\ensuremath{\mathtt{SL}}}              % 特殊线性群 special linear group
\newcommand{\SO}{\ensuremath{\mathtt{SO}}}              % 特殊正交群 special orthogonal group
\newcommand{\SU}{\ensuremath{\mathtt{SU}}}              % 特殊幺正群 special unitary group
\newcommand{\TO}{\ensuremath{\mathtt{O}}}               % 正交群 orthogonal group
\newcommand{\TU}{\ensuremath{\mathtt{U}}}               % 幺正群 unitary group


% 使用 mathbb 字族的符号定义
\newcommand{\B}{\ensuremath{\mathbb{B}}}                % 指闭单位球
\newcommand{\C}{\ensuremath{\mathbb{C}}}                % 指复数域
\newcommand{\D}{\ensuremath{\mathbb{D}}}                % 复平面上的单位圆盘
\newcommand{\F}{\ensuremath{\mathbb{F}}}                % 泛指一般的数域或用于有限域\F_n中
\renewcommand{\H}{\ensuremath{\mathbb{H}}}              % Hamilton四元数除环
\newcommand{\J}{\ensuremath{\mathbb{J}}}                % 无理数集(非标准符号)
\newcommand{\K}{\ensuremath{\mathbb{K}}}                % 实数域或复数域
\newcommand{\N}{\ensuremath{\mathbb{N}}}
\newcommand{\Q}{\ensuremath{\mathbb{Q}}}                % 有理数域
\newcommand{\R}{\ensuremath{\mathbb{R}}}                % 实数域
\renewcommand{\S}{\ensuremath{\mathbb{S}}}              % 用于球面S^n
\newcommand{\T}{\ensuremath{\mathbb{T}}}                % 用于环面T^n
\newcommand{\Z}{\ensuremath{\mathbb{Z}}}                % 整数环
\newcommand{\Zo}{\ensuremath{\mathbb{Z}_{\geqslant 0}}} % 非负整数集
\newcommand{\Zi}{\ensuremath{\mathbb{Z}_{\geqslant 1}}} % 正整数集


% 使用 mathcal 字族的符号定义

\newcommand{\CA}{\mathcal{A}}
\newcommand{\CB}{\mathcal{B}}
\newcommand{\CC}{\mathcal{C}}
\newcommand{\CD}{\mathcal{D}}
\newcommand{\CE}{\mathcal{E}}
\newcommand{\CF}{\mathcal{F}}
\newcommand{\CG}{\mathcal{G}}
\newcommand{\CH}{\mathcal{H}}
\newcommand{\CK}{\mathcal{K}}
\newcommand{\CL}{\mathcal{L}}
\newcommand{\CM}{\mathcal{M}}
\newcommand{\CN}{\mathcal{N}}
\newcommand{\CS}{\mathcal{S}}
\newcommand{\CT}{\mathcal{T}}
\newcommand{\CU}{\mathcal{U}}

\newcommand{\Fs}{\ensuremath{\CF_{\sigma}}}             % F_sigma 集
\newcommand{\Gd}{\ensuremath{\CG_{\delta}}}             % G_delta 集
\newcommand{\Fr}{\ensuremath{\CF_{r}}}                  % 有限秩算子全体


% 使用 mathfrak 字族的符号定义

\newcommand{\FB}{\mathfrak{B}}                          % Borel代数
\newcommand{\FL}{\mathfrak{L}}                          % Lebesgue可测代数